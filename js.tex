\documentclass{beamer}
\usepackage[ngerman]{babel}
%\usepackage[T1]{fontenc}
\usepackage[utf8]{inputenc}
\usepackage{wasysym}
\usepackage{listings}
\usepackage[]{hyperref}

% From: http://lenaherrmann.net/2010/05/20/javascript-syntax-highlighting-in-the-latex-listings-package
\lstdefinelanguage{JavaScript}{
  keywords={typeof, new, true, false, catch, function, return, null, catch, switch, var, if, in, while, do, else, case, break},
  keywordstyle=\color{blue}\bfseries,
  ndkeywords={class, export, boolean, throw, implements, import, this},
  ndkeywordstyle=\color{darkgray}\bfseries,
  identifierstyle=\color{black},
  sensitive=false,
  comment=[l]{//},
  morecomment=[s]{/*}{*/},
  commentstyle=\color{purple}\ttfamily,
  stringstyle=\color{red}\ttfamily,
  morestring=[b]',
  morestring=[b]"
}

% Ein minimalistisches Layout
\lstset{language=Javascript,basicstyle=\small,stringstyle=\ttfamily,keywordstyle={[2]\small}}

% Farb- und sonstige Gestaltung der Folien
\usetheme{Dresden}
\usefonttheme{serif}
%\useinnertheme{circles}
\usecolortheme{beaver}
\setbeamercolor{itemize item}{fg=darkred}
\setbeamercolor{itemize subitem}{fg=black}
\setbeamertemplate{itemize item}[>]
\setbeamertemplate{itemize subitem}[-]

% Keine Navigationsleiste am unteren Folienrand
\beamertemplatenavigationsymbolsempty

% Joa, was so an Informationen zu einer Präsentation gehört
\author{Julius Roob}
\title{Javascript}
\subtitle{Lua lernen \checked\\Entwickeln üben \checked\\IRC Bot haben \checked}
\keywords{Javascript, JQuery}
\institute{Chaos inKL.}
\date{}

\begin{document}
	\section{Einleitung}
		\begin{frame}
			\titlepage
		\end{frame}
	
		\begin{frame}{Übersicht}
			\tableofcontents
		\end{frame}
	
	
	\section{Geschichte}
		\begin{frame}{Geschichte}{}
			\begin{itemize}
				\item Netscape
				\item JAVAscript
				\item Microsoft
				\item ECMAScript
				\item DOM
			\end{itemize}
		\end{frame}

		\begin{frame}{Datentypen}{}
			\begin{itemize}
				\item Strings
				\item Objekte
				\item Arrays
				\item null
				\item undefined
				\item Zahlen
					\begin{itemize}
						\item $(NEGATIVE|POSITIVE)\_INFINITY$
						\item $(MIN|MAX)\_VALUE$
						\item NaN
					\end{itemize}
			\end{itemize}
		\end{frame}

		\begin{frame}[fragile,shrink=5]{Syntax}
			\begin{itemize}
				\item Funktionen
				\begin{lstlisting}
					> function f(x) { return x + 1; }
				\end{lstlisting}
				\item Variablen
				\begin{lstlisting}
					> var f = function(x) { return x + 2; };
				\end{lstlisting}
				\item \textbf{foreach}
					\begin{lstlisting}
						> for(var i = 0; i < array.length; i++)
					\end{lstlisting}
					\begin{lstlisting}
						> for(<var index> in <array>){}
					\end{lstlisting}
					\begin{lstlisting}
						> Array.forEach(f)
					\end{lstlisting}
			\end{itemize}
		\end{frame}
		
		\begin{frame}[fragile,shrink=5]{Closures}
			\begin{lstlisting}
				> function a(x) {
				>   return function() {
				>     return x++;
				>   }
				> }
				> x = a(100)
				> x()
				100
				> x()
				101
				> x()
				102
			\end{lstlisting}
		\end{frame}
		
		\begin{frame}[fragile,shrink=5]{Prototyping}
			\lstset{numbers=none}
			\begin{lstlisting}[title={Schreibweise}]
				function <Name>(<Parameter>) ... end
				-- ist das gleiche wie:
				<Name> = function(<Parameter>) ... end
			\end{lstlisting}

			\begin{lstlisting}[title={Mehrere Rückgabewerte}]
				local a, b, c = unpack({"hallo", "welt", "!"})

				print(a, b, unpack({"!", "wie geht es dir?"}))
			\end{lstlisting}
			
			\begin{columns}
				\column{.50\textwidth}
					\begin{lstlisting}[title={OOP}]
						welt:hallo("!")
						-- ist kurz fuer:
						welt.hallo(welt, "!")
					\end{lstlisting}
				
				\column{.50\textwidth}
					\begin{lstlisting}[title={Proper Tail Calls}]
						function f (x)
						  return g(x)
						end
					\end{lstlisting}
			\end{columns}
		\end{frame}
		
		\begin{frame}[fragile]{Sandboxing}
			Sehr komfortabel in Lua.

			\begin{description}
				\item[Scope] \verb+setfenv()+
				\item[Prozessor] \verb+function e() error("CPU limit reached!") end+ \verb+debug.sethook(e, "", 1000)+
				\item[Speicher] \verb+setrlimit()+ und \verb+pcall()+
			\end{description}
		\end{frame}
		
		\begin{frame}[fragile]{OOP}
			\lstset{numbers=left}
			Alternative zur bereits beschriebenen OOP: \textit{Closures for Scopes}
			\begin{lstlisting}
			function factory()
			  local self = {kaffee=0}
			  local interface = {}
			  function interface.set(n)
			    self.kaffee = n
			  end
			  function interface.get()
			    return self.kaffee
			  end
			  return interface
			end
			\end{lstlisting}
		\end{frame}

		\begin{frame}[fragile]{Coroutines}
			\begin{columns}
				\column{.50\textwidth}
					\begin{lstlisting}
						function test()
						  local a = 0
						  while true do
						    a = a + 1
						    coroutine.yield(a)
						  end
						end
					\end{lstlisting}
				\column{.50\textwidth}
					\begin{lstlisting}
						> a = coroutine.create(test)
						> print(coroutine.resume(a))
						true	1
						> print(coroutine.resume(a))
						true	2
						> print(coroutine.resume(a))
						true	3
						> print(coroutine.resume(a))
						true	4
						> print(coroutine.resume(a))
						true	5
					\end{lstlisting}
			\end{columns}
		\end{frame}
		
		\begin{frame}[fragile]{Luarocks}
			\lstset{numbers=none}
			Einfaches installieren von zusätzlichen Bibliotheken bzw. Paketen.
			\begin{lstlisting}
				# luarocks search pcre
				# luarocks install lrexlib-pcre
			\end{lstlisting}
			\vskip15pt
			Kann von einem Benutzer im Homedir installiert werden.
			\vskip15pt
			Skript mit Rocks ausführen:

			\begin{lstlisting}
				# lua -lluarocks.loader aurora.lua
			\end{lstlisting}
		\end{frame}
		
		\begin{frame}[fragile]{JSON}
			Einfach und schön

			\begin{lstlisting}
				{
				  "wasser":{"julius":1},
				  "africola":{"bnerd":1,"julius":0},
				  "baileys":{"julius":1},
				  "cappu":{"lucy":2,"julius":7,"mullej":7},
				  "mate":{"julius":0},
				  "kaffee":{"julius":2,"lastofthewolves":1}
				}
			\end{lstlisting}
			\verb+json.encode()+ und \verb+json.decode()+ sind wohl die einfachste Möglichkeit, Tables zu serialisieren.
		\end{frame}
		
		\begin{frame}[fragile,shrink=5]{Copas}{Coroutine Oriented Portable Asynchronous Services for Lua}
			\lstset{numbers=left}

			\begin{lstlisting}[title={Primitiver Chatserver},firstnumber=3,frame=single]
				users = {}
				function handle(client)
				  copas.send(client, "name? ")
				  local name = copas.receive(client, "*l")
				  users[name] = client
				  while true do
				    local line = copas.receive(client, "*l")
				    for _,sock in pairs(users) do
				      copas.send(sock, name .. ": " .. line .. "\n")
				    end
				  end
				end
				serv = socket.bind("0.0.0.0", 1025)
				copas.addserver(serv, handle)
				copas.loop()
			\end{lstlisting}
		\end{frame}
	
	
	\section{IRC}
		\begin{frame}{IRC}{Inhalt}
			\begin{itemize}
				\item Kurze Beschreibung 
				\item Nachrichtenformat
				\item Bleistifte
				\item CTCP
			\end{itemize}
		\end{frame}
		
		\begin{frame}{Kurze Beschreibung (aka Wikipedia lite)}
			\begin{itemize}
				\item Gibt es als \textbf{Internet} Relay Chat seit 1988.
				\item Es existiert eine unüberschaubare Anzahl an Clients, Bots und Servern
				\item Thin Clients
			\end{itemize}
		\end{frame}
		
		\begin{frame}[fragile,shrink=5]{Nachrichtenformat}
			Die Kommunikation zwischen Server und Client findet ausschließlich durch max. 510 Byte lange mit einem CRLF getrennte Nachrichten statt.
			
			\begin{lstlisting}[language=,breaklines=true,breakatwhitespace=true]
				NICK julius_
				USER julius julius irc.hackint.org :Julius
				:irc.hackint.org NOTICE AUTH :*** No Ident response
				:irc.hackint.org NOTICE AUTH :*** Couldn't look up your hostname
				:irc.hackint.org 001 julius :Welcome to the hackint Internet Relay Chat Network julius
			\end{lstlisting}
			
			\begin{lstlisting}[language=,breaklines=true,breakatwhitespace=true]
				PING LAG2126774742
				:irc.hackint.org PONG irc.hackint.org :LAG2126774742
				PRIVMSG julius :blub!
				:irc.hackint.org 301 julius_ julius :auf nach hause
			\end{lstlisting}
			
			\begin{lstlisting}[language=,breaklines=true,breakatwhitespace=true]
				:eBrnd!~eBrnd@ein.anderer.server PRIVMSG #c3kl :bitte irgendwas :)
			\end{lstlisting}
		\end{frame}

		\begin{frame}[fragile]{Bleistife}
			\begin{itemize}
				\item \verb+PRIVMSG <Channel> <Nachricht>+
				\item \verb+JOIN <Channel>(,<Channel>)*+
				\item \verb+QUIT <Nachricht>+
				\item \verb+NICK <Name>+
			\end{itemize}
		\end{frame}

		\begin{frame}[fragile,shrink=5]{CTCP}{Client-To-Client Protocol}
			\begin{lstlisting}[language=,breaklines=true,breakatwhitespace=true,title={Format}]
				"\001<Befehl> <Parameter>\001"
			\end{lstlisting}

			\begin{lstlisting}[language=,breaklines=true,breakatwhitespace=true,title={Beispiele}]
				\001ACTION demonstriert mal eben CTCP Action\001
				-> 10:10  *julius demonstriert mal eben CTCP ACTION

				\001VERSION\001
				-> 10:16 [freenode] CTCP VERSION reply from julius: irssi v0.8.12

				\001DCC SEND chess.lua 2213957892 48042 1682\001
			\end{lstlisting}
		\end{frame}
	
	\section{Aurora}
		\begin{frame}{Aurora}{Inhalt}
			\begin{itemize}
				\item Motivation
				\item Architektur
				\item Module
				\item Probleme
			\end{itemize}
		\end{frame}

		\begin{frame}{Motivation}
			Warum einen eigenen IRC Bot schreiben?

			\begin{itemize}
				\item Sprache Lua lernen
				\item Eventbasierende Socket IO verinnerlichen
				\item IRC Bots sind immer praktisch
				\pause
				\item Spaß
			\end{itemize}
		\end{frame}

		\begin{frame}{Architektur}
			Uh - ja...
			\begin{itemize}
				\item Module
				\item ???
				\item IRC Netzwerke
			\end{itemize}
		\end{frame}

		\begin{frame}{Module}
			\begin{itemize}
				\item Konstruktor
				\item Destruktor
				\item Step
				\item Handler
					\begin{itemize}
						\item DISCONNECT
						\item CONNECT
						\item PRIVMSG
						\item ... (siehe RFC :)
					\end{itemize}
			\end{itemize}
		\end{frame}
		
		\begin{frame}{Probleme}
			\begin{itemize}
				\item Module organisieren?
				\item Zwischen Modulen und Netzwerken herrscht Anarchie
				\item Authentifizierung im IRC ist unschön
			\end{itemize}
		\end{frame}

		\begin{frame}{Fragen?}{Vorschläge, Anregungen...}
			\url{http://www.github.com/Zottel/Aurora}

			\textbf{:-)}
		\end{frame}

		\begin{frame}{Quellen}
			\begin{itemize}
				\item \url{http://de.selfhtml.org/javascript/intro.htm}
				\item \url{http://matthiasschuetz.com/}
				\item Google Tech Talk: "Javascript: The Good Parts"
					(\url{http://www.youtube.com/watch?v=hQVTIJBZook})
				\item \url{http://inimino.org/~inimino/blog/javascript_semicolons}
				\item ChaosRadio Expres: Node.js
				\item \url{http://shichuan.github.com/javascript-patterns/}
			\end{itemize}
		\end{frame}
		
\end{document}

